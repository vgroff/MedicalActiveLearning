Bounding box and Image-specific fine-tuning-based segmentation (BIFSeg) is a framework proposed by Wang et al. \cite{BIFSeg} which combines Graph Cuts with a CNN to allow interactive segmentation. The user first draws a bounding box around the object of interest, so as to simplify the segmentation process. The CNN is based on the P-Net architecture described in section \ref{DeepIGeoS} above, which outputs for each pixel a likelihood that the pixel is part of the object. The cost function being minimized is of the same form as Eq. \ref{eq:costFunc}, but the likelihood used in the calculation of the regional terms is that which is produced by the CNN, instead of being calculated from the histogram. The CNN and CRF produce and initial estimation, and the user can then label pixels correctly, particularly relabeling mislabeled pixels. The CNN is then fine-tuned on a per-image basis using the user input. The training progresses in two stages: a label update state, where the network parameters $\boldsymbol{\theta}$ are set and the binary label array $\mathbf{Y}$ is calculated (using graph cuts as described in section \ref{graphCuts}), and a network update set where a new $\boldsymbol{\theta}$ is learned from the $\mathbf{Y}$ calculated in the previous step (using backpropagation).

For the network update state, cost functions of pixels which are user provided are multiplied by some constant $w$ to boost their relative importance to the cost function. Pixels which have an output near to $0.5$ are set to have a weight of 0, and similarly with pixels within a certain geodesic distance (as explained in section \ref{DeepIGeoS}) of user inputs, since they are likely to have been misclassified by the CNN originally. The intuitition here is that these pixels should change by proxy of the learning that the scribbles undergo. This means additional hyper parameters need to be defined: $t_0$ and $t_1$, the cut-off points either side of $0.5$, $\sigma$ and $l$ for the geodesic distance calculation (see Eq \ref{eq:geodesic}), the maximal geodesic length $\epsilon$, as well as $w$. Wang et al. do not talk about the hyper parameters in the geodesic calculation, and from that it seems likely that they set $\sigma$ to be the same value as in the boundary term equation in the CRF (Eq. \ref{eq:boundTerm}) and $l=0$. 

Wang et al. also save the output from the part of the network that is not being trained during the fine-tuning process, which makes for faster inference during the training process. 